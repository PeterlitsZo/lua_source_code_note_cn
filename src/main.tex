\documentclass[10pt, b5paper]{ctexbook}


% Third-part package
%%%%%%%%%%%%%%%%%%%%%%%%%%%%%%%%%%%%%%%%%%%%%%%%%%%%%%%%%%%%%%%%%%%%%%%%%%%%%%%
\usepackage{listings}
\usepackage[right=1.8cm, left=2.2cm, top=2.4cm, bottom=2.4cm]{geometry}
\usepackage{xcolor}


% The base settings
%%%%%%%%%%%%%%%%%%%%%%%%%%%%%%%%%%%%%%%%%%%%%%%%%%%%%%%%%%%%%%%%%%%%%%%%%%%%%%%

% for listings
\lstset{
    basicstyle=\ttfamily\fontsize{7.2pt}{7.2pt}\selectfont,
    breaklines,
    frame=tRBl,
    xleftmargin=2.5em,
    commentstyle=\color{black!50!white},
    emphstyle=\bfseries,
    stringstyle=\color{black!75!white},
}

% set the mono font
\setmonofont{inconsolata}


% Command for the note
%%%%%%%%%%%%%%%%%%%%%%%%%%%%%%%%%%%%%%%%%%%%%%%%%%%%%%%%%%%%%%%%%%%%%%%%%%%%%%%

% use cmd command to input the command, I do not like to use verb in `inline`
% mode.
\newcommand{\cmd}[1]{\,{\tt #1}\,}

% use verbatim to input the code
\lstnewenvironment{code}[2]{\lstset{caption={[#2]#2}, label={#1}}}{}
\lstnewenvironment{ccode}[2]{\lstset{language=C, caption={[#2]#2}, label={#1}}}{}
\lstnewenvironment{luacode}[2]{\lstset{language=lua, caption{[#2]#2}, label={#1}}}{}


% the main part
%%%%%%%%%%%%%%%%%%%%%%%%%%%%%%%%%%%%%%%%%%%%%%%%%%%%%%%%%%%%%%%%%%%%%%%%%%%%%%%

\title{实现语言!基于Lua}

\begin{document}
    \maketitle

    
%%%%%%%%%%%%%%%%%%%%%%%%%%%%%%%%%%%%%%%%%%%%%%%%%%%%%%%%%%%%%%%%%%%%%%%%%%%%%%%

\chapter{序}

这是一本关于\cmd{lua}如何实现的文章。本书的源代码可以访问托管在\cmd{github}上的
仓库,即\cmd{github.com/PeterlitsZo/lua\_source\_code\_note\_cn}。

\section{本书读者}

我们希望读者有着较强的\cmd{C}语言使用经验,对基本的计算机术语有着基础的认识(\emph{%
如:状态机,面向对象等})。如果不熟悉的话,推荐阅读《C和指针》或者《The C Programming
Language》。

此外重要的一点是了解编译原理,当然,如果读者之前完全没有尝试了解编译原理的,我也
会尽量在探讨\cmd{lua}语言的时候讲得通俗易懂。

本书不要求读者会使用\cmd{lua}。我们的目的是寻找如何实现一门编程语言的具体示例。

我们希望读者使用过UNIX shell的经历。

\section{为什么是\cmd{lua}}

\cmd{lua}是一个简洁、轻量、对\cmd{C}友好的脚本语言。它的实现非常优雅,是一个有着
属于它独特魅力的语言。

\section{\cmd{lua}版本}

本书采用的\cmd{lua}是正在开发中的版本,你可以在\cmd{github.com}中运行命令进行克隆:
\begin{code}{about[clone-lua]}{在\cmd{github.com}克隆到本地}
$ git clone github.com/lua/lua.git ~/lua
$ cd ~/lua
$ git reset --hard 6bc0f135
\end{code}

运行了相关的shell代码\ref{about[clone-lua]},我们可以说你现在看到的\cmd{lua}代码
和编写本书时看到的\cmd{lua}代码基本一致。

本书的目的并非\cmd{lua}代码是如何实现的(\emph{但是也正如所示,我们的的确确会探讨
\cmd{lua}是如何实现的}),我们关注的重点更是如何从零开始打造一个图灵完备的编程语言。
这也是我们为什么不强求\cmd{lua}的实现版本,我们关注的是一个语言如何从零实现的。

\section{致谢}

本书有很多部分都是基于源代码、维基百科和\cmd{lua}的官方网站\footnote{见:\cmd{lua.prg}%
。},在这里对开源组织表示感谢。

 % about the book

    \tableofcontents
    \newpage

    
%%%%%%%%%%%%%%%%%%%%%%%%%%%%%%%%%%%%%%%%%%%%%%%%%%%%%%%%%%%%%%%%%%%%%%%%%%%%%%%

\chapter{Lex 分词}

运行一个解释器编程语言,我们一般会使用\cmd{lex}来对程序进行分词,而使用\cmd{yyac}%
来对标记流(\cmd{lex}分完词之后的结果)生成抽象语法树。

在分析之前,我们先在\cmd{lua.c}中根据\cmd{main}函数来看看它如何解析命令行参数。

\section{命令行解析}

我们在了解如何分词之前我们需要了解它到底在什么时候开始分词的。在一开始我们跟着
调用顺序从\cmd{main}入手,随后是\cmd{pmain}。

\subsection{\cmd{main}入口}

我们可以看到如代码\ref{lua.c::main}所示一样,

\begin{ccode}{lua.c::main}{\cmd{lua.c}中的\cmd{main}函数}
int main (int argc, char **argv) {
  int status, result;
  lua_State *L = luaL_newstate();  /* create state */
  if (L == NULL) {
    l_message(argv[0], "cannot create state: not enough memory");
    return EXIT_FAILURE;
  }
  lua_pushcfunction(L, &pmain);  /* to call 'pmain' in protected mode */
  lua_pushinteger(L, argc);  /* 1st argument */
  lua_pushlightuserdata(L, argv); /* 2nd argument */
  status = lua_pcall(L, 2, 1, 0);  /* do the call */
  result = lua_toboolean(L, -1);  /* get result */
  report(L, status);
  lua_close(L);
  return (result && status == LUA_OK) ? EXIT_SUCCESS : EXIT_FAILURE;
}
\end{ccode}

我们首先创建了一个\cmd{lua\_state*}对象来存储\cmd{VM}\footnote{\cmd{VM},可以说是
程序虚拟机的意思。}的状态。

之后我们将临时函数(这里是\cmd{pmain})和命令行变量入栈调用,最后退出。

\subsection{\cmd{pmain}函数}


 % lexer
    
%%%%%%%%%%%%%%%%%%%%%%%%%%%%%%%%%%%%%%%%%%%%%%%%%%%%%%%%%%%%%%%%%%%%%%%%%%%%%%%

\chapter{结构}

在这一章节,我们会来研究包括:\cmd{lua}的状态机,即\cmd{lua\_State* L}的相关细节。

\section{状态机\cmd{lua\_State}细节}

在代码\ref{lua.c::main}中我们使用\cmd{luaL\_newState}来初始化一个\cmd{lua\_State*}%
指针,目前问题就是:\cmd{lua\_State}中到底储存了什么?

根据自动机的理论,我们可以使用一个有限状态机来让语言图灵完备,那很显然这个有限状
态机就正好被\cmd{lua\_State*}类型的指针所指向的,而

[TODO]

 % data struct
\end{document}

