
%%%%%%%%%%%%%%%%%%%%%%%%%%%%%%%%%%%%%%%%%%%%%%%%%%%%%%%%%%%%%%%%%%%%%%%%%%%%%%%

\chapter*{序}

这是一本关于\cmd{lua}如何实现的文章。本书的源代码可以访问托管在\cmd{github}上的
仓库,即\cmd{github.com/PeterlitsZo/lua\_source\_code\_note\_cn}。

\subsection*{本书读者}

我们希望读者有着较强的\cmd{C}语言使用经验,对基本的计算机术语有着基础的认识(\emph{%
如:状态机,面向对象等})。如果不熟悉的话,推荐阅读《C和指针》或者《The C Programming
Language》。

此外重要的一点是了解编译原理,当然,如果读者之前完全没有尝试了解编译原理的,我也
会尽量在探讨\cmd{lua}语言的时候讲得通俗易懂。

本书不要求读者会使用\cmd{lua}。我们的目的是寻找如何实现一门编程语言的具体示例。

我们希望读者使用过UNIX shell的经历。


\subsection*{为什么是\cmd{lua}}

\cmd{lua}是一个简洁、轻量、对\cmd{C}友好的脚本语言。它的实现非常优雅,是一个有着
属于它独特魅力的语言。


\subsection*{\cmd{lua}版本}

本书采用的\cmd{lua}是正在开发中的版本,你可以在\cmd{github.com}中运行命令进行克隆:
\begin{code}{about[clone-lua]}{从\cmd{github.com}克隆到本地}
$ git clone github.com/lua/lua.git ~/lua
$ cd ~/lua
$ git reset --hard 6bc0f135
\end{code}

\def\useless$

运行了相关的shell代码\ref{about[clone-lua]},我们可以说你现在看到的\cmd{lua}代码
和编写本书时看到的\cmd{lua}代码基本一致。

本书的目的与其说是观察\cmd{lua}代码是如何实现的,还不如说是为了观察如何从零开始打
造一个图灵完备的编程语言。而这也正是我们为什么不强求\cmd{lua}的实现版本的一个原因。


\subsection*{本书构架}

在第一章,我们探寻\cmd{lex}框架:它将字符串转换为有意义的字符串流(token stream),
在此之前我们还会以极快的速度从\cmd{main}函数通过\cmd{REPL}途径走到\cmd{lex}中。

在第二章,我们会探寻它的如何从\cmd{lex}产生的字符串流转换为抽象树。

在第三章,我们会探索\cmd{lua}的一些数据结构的表达方式。


\subsection*{勘误}

如果有任何错误的话,请访问\cmd{github.com/PeterlitsZo/lua\_source\_code\_note\_cn},
提交\cmd{issue}或者申请\cmd{Pull request}即可。

也欢迎加入来完善本书。


\subsection*{致谢}

本书有很多部分都是基于源代码、维基百科和\cmd{lua}的官方网站\footnote{见:\cmd{lua.prg}%
。},在这里对开源组织表示感谢,如果这本书产生收益的话,其中一半会支付给\cmd{lua}组织以
支撑它们接下来的运行。

